\documentclass{bduMasters}

\begin{document}
\tableofcontents
\chapter{ONE}
\section{Hello}
The layout package provides a very convenient solution to visualizing the document's current layout—and the values of various LaTeX parameters which determine that layout. It provides two commands: layout and layout* which draw a graphic representing the current layout. The starred version (layout*) recalculates the internal values used to draw the graphic, which can be useful if you make changes to LaTeX's page-layout parameters.
The layout package provides a very convenient solution to visualizing the document's current layout—and the values of various LaTeX parameters which determine that layout. It provides two commands: layout and layout* which draw a graphic representing the current layout. The starred version (layout*) recalculates the internal values used to draw the graphic, which can be useful if you make changes to LaTeX's page-layout parameters.
The layout package provides a very convenient solution to visualizing the document's current layout—and the values of various LaTeX parameters which determine that layout. It provides two commands: layout and layout* which draw a graphic representing the current layout. The starred version (layout*) recalculates the internal values used to draw the graphic, which can be useful if you make changes to LaTeX's page-layout parameters.
The layout package provides a very convenient solution to visualizing the document's current layout—and the values of various LaTeX parameters which determine that layout. It provides two commands: layout and layout* which draw a graphic representing the current layout. The starred version (layout*) recalculates the internal values used to draw the graphic, which can be useful if you make changes to LaTeX's page-layout parameters.
The layout package provides a very convenient solution to visualizing the document's current layout—and the values of various LaTeX parameters which determine that layout. It provides two commands: layout and layout* which draw a graphic representing the current layout. The starred version (layout*) recalculates the internal values used to draw the graphic, which can be useful if you make changes to LaTeX's page-layout parameters.

\chapter{TWO}
\section{Hello}
The layout package provides a very convenient solution to visualizing the document's current layout—and the values of various LaTeX parameters which determine that layout. It provides two commands: layout and layout* which draw a graphic representing the current layout. The starred version (layout*) recalculates the internal values used to draw the graphic, which can be useful if you make changes to LaTeX's page-layout parameters. Here is an example:
The layout package provides a very convenient solution to visualizing the document's current layout—and the values of various LaTeX parameters which determine that layout. It provides two commands: layout and layout* which draw a graphic representing the current layout. The starred version (layout*) recalculates the internal values used to draw the graphic, which can be useful if you make changes to LaTeX's page-layout parameters.
The layout package provides a very convenient solution to visualizing the document's current layout—and the values of various LaTeX parameters which determine that layout. It provides two commands: layout and layout* which draw a graphic representing the current layout. The starred version (layout*) recalculates the internal values used to draw the graphic, which can be useful if you make changes to LaTeX's page-layout parameters.
The layout package provides a very convenient solution to visualizing the document's current layout—and the values of various LaTeX parameters which determine that layout. It provides two commands: layout and layout* which draw a graphic representing the current layout. The starred version (layout*) recalculates the internal values used to draw the graphic, which can be useful if you make changes to LaTeX's page-layout parameters.

\chapter{THREE}
\section{Hello}
The layout package provides a very convenient solution to visualizing the document's current layout—and the values of various LaTeX parameters which determine that layout. It provides two commands: layout and layout* which draw a graphic representing the current layout. The starred version (layout*) recalculates the internal values used to draw the graphic, which can be useful if you make changes to LaTeX's page-layout parameters. Here is an example:
The layout package provides a very convenient solution to visualizing the document's current layout—and the values of various LaTeX parameters which determine that layout. It provides two commands: layout and layout* which draw a graphic representing the current layout. The starred version (layout*) recalculates the internal values used to draw the graphic, which can be useful if you make changes to LaTeX's page-layout parameters.
The layout package provides a very convenient solution to visualizing the document's current layout—and the values of various LaTeX parameters which determine that layout. It provides two commands: layout and layout* which draw a graphic representing the current layout. The starred version (layout*) recalculates the internal values used to draw the graphic, which can be useful if you make changes to LaTeX's page-layout parameters.
The layout package provides a very convenient solution to visualizing the document's current layout—and the values of various LaTeX parameters which determine that layout. It provides two commands: layout and layout* which draw a graphic representing the current layout. The starred version (layout*) recalculates the internal values used to draw the graphic, which can be useful if you make changes to LaTeX's page-layout parameters.
The layout package provides a very convenient solution to visualizing the document's current layout—and the values of various LaTeX parameters which determine that layout. It provides two commands: layout and layout* which draw a graphic representing the current layout. The starred version (layout*) recalculates the internal values used to draw the graphic, which can be useful if you make changes to LaTeX's page-layout parameters.
The layout package provides a very convenient solution to visualizing the document's current layout—and the values of various LaTeX parameters which determine that layout. It provides two commands: layout and layout* which draw a graphic representing the current layout. The starred version (layout*) recalculates the internal values used to draw the graphic, which can be useful if you make changes to LaTeX's page-layout parameters.
The layout package provides a very convenient solution to visualizing the document's current layout—and the values of various LaTeX parameters which determine that layout. It provides two commands: layout and layout* which draw a graphic representing the current layout. The starred version (layout*) recalculates the internal values used to draw the graphic, which can be useful if you make changes to LaTeX's page-layout parameters.

\section{Nathan}
The layout package provides a very convenient solution to visualizing the document's current layout—and the values of various LaTeX parameters which determine that layout. It provides two commands: layout and layout* which draw a graphic representing the current layout. The starred version (layout*) recalculates the internal values used to draw the graphic, which can be useful if you make changes to LaTeX's page-layout parameters. Here is an example:
The layout package provides a very convenient solution to visualizing the document's current layout—and the values of various LaTeX parameters which determine that layout. It provides two commands: layout and layout* which draw a graphic representing the current layout. The starred version (layout*) recalculates the internal values used to draw the graphic, which can be useful if you make changes to LaTeX's page-layout parameters.
The layout package provides a very convenient solution to visualizing the document's current layout—and the values of various LaTeX parameters which determine that layout. It provides two commands: layout and layout* which draw a graphic representing the current layout. The starred version (layout*) recalculates the internal values used to draw the graphic, which can be useful if you make changes to LaTeX's page-layout parameters.
The layout package provides a very convenient solution to visualizing the document's current layout—and the values of various LaTeX parameters which determine that layout. It provides two commands: layout and layout* which draw a graphic representing the current layout. The starred version (layout*) recalculates the internal values used to draw the graphic, which can be useful if you make changes to LaTeX's page-layout parameters.
The layout package provides a very convenient solution to visualizing the document's current layout—and the values of various LaTeX parameters which determine that layout. It provides two commands: layout and layout* which draw a graphic representing the current layout. The starred version (layout*) recalculates the internal values used to draw the graphic, which can be useful if you make changes to LaTeX's page-layout parameters.
The layout package provides a very convenient solution to visualizing the document's current layout—and the values of various LaTeX parameters which determine that layout. It provides two commands: layout and layout* which draw a graphic representing the current layout. The starred version (layout*) recalculates the internal values used to draw the graphic, which can be useful if you make changes to LaTeX's page-layout parameters.
The layout package provides a very convenient solution to visualizing the document's current layout—and the values of various LaTeX parameters which determine that layout. It provides two commands: layout and layout* which draw a graphic representing the current layout. The starred version (layout*) recalculates the internal values used to draw the graphic, which can be useful if you make changes to LaTeX's page-layout parameters.

\end{document}


%%% Local Variables:
%%% mode: latex
%%% TeX-master: t
%%% End:
